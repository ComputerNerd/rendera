\documentclass{article}

\title{Automatic Color Restoration}

\author{Joe Davisson} %% \texttt{joe_7@sbcglobal.net}}

\usepackage{amsmath}

\begin{document}
\maketitle

A simple algorithm is presented for automatically correcting the
color and contrast in faded or otherwise undesirable images.
Because often the exact cause of the discoloration is unknown, this
algorithm is based on correcting the problems of color-cast,
contrast, and saturation loss individually.

\section{Color-Cast Removal}

One way to correct color-cast is to create adjustment factors from an
average of all the pixels in an image:

\[ R_{adjust} = (255 - R_{average}) \]
\[ G_{adjust} = (255 - G_{average}) \]
\[ B_{adjust} = (255 - B_{average}) \]

These are then averaged with each pixel in the image:

\[ R_{new} = (R_{old} + R_{adjust}) / 2 \]
\[ G_{new} = (G_{old} + G_{adjust}) / 2 \]
\[ B_{new} = (B_{old} + B_{adjust}) / 2 \]

While this corrects the colors cast, the image must be manually adjusted
to restore contrast and saturation. We aim to do these automatically.

\section{Contrast Correction}

The means to restore constrast may be included in the adjustment factors if
they are changed to power functions instead of simply a color to be averaged.
The square root ensures that over-correction is not possible, preventing
blown-out highlights in the restored image.

\[ R_{adjust} = (256 / (256 - R_{average})) / sqrt(256 / (R_{average} + 1)) \]
\[ G_{adjust} = (256 / (256 - G_{average})) / sqrt(256 / (G_{average} + 1)) \]
\[ B_{adjust} = (256 / (256 - B_{average})) / sqrt(256 / (B_{average} + 1)) \]

These are then applied to the image the source image:

\[ R_{new} = 255 * pow(R_{old} / 255, R_{adjust}) \]
\[ G_{new} = 255 * pow(G_{old} / 255, G_{adjust}) \]
\[ B_{new} = 255 * pow(B_{old} / 255, B_{adjust}) \]

\section{Saturation Equalization}

We make the assumption that the most saturated color in the original faded
image should be pushed to the maximum value, and other saturation levels pushed in relation to that. This is similar to histogram equalization, which is also
well-known and easy to implement.

Histogram equalization is performed on saturation information only, with the option of preserving the luminance of the original image. The latter is important to ensure perceptual integrity.

\end{document}

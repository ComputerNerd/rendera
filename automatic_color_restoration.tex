\documentclass{article}

\title{Automatic Color Restoration}

\author{Joe Davisson} %% \texttt{joe_7@sbcglobal.net}}

\usepackage{amsmath}

\begin{document}
\maketitle

A simple algorithm is presented for automatically correcting the
color and contrast in faded images. Because the exact nature or cause
of the discoloration is often unknown, a more general approach is pursued.

There are at least three contributors to an undesirable image. These include the problems of color cast, contrast, and saturation loss. With these problems corrected, most images will appear "normal" to the eye.

\section{Color Cast Removal}
Perhaps the simplest way to correct color cast is to simply average each
pixel by the inverted average of all the pixels:

\[ C_{new} = \frac{C_{old} + \left(255 - C_{average}\right) }{2} \]

While this serves to reverse the the color cast, the image must be manually
adjusted to restore contrast and saturation. Therefore we aim to do these
adjustments automatically.

\section{Contrast Correction}
Color and constrast correction may be combined together into a single
adjustment factor. This is accomplished with the following formula:

\[ C_{adjust} = \frac{\left(\frac{256} {256 - C_{average}}\right)}
                     {\sqrt{\frac{256} { C_{average} + 1}}} \]

The result is an exponent which is applied to each pixel in the original image.
(Employing a square root in the denominator prevents over-correction in the
restored image.) The adjustment factors are applied as follows:

\[ C_{new} = 255 * \left(\frac{C_{old}}{255}\right) ^ {C_{adjust}} \]

\section{Saturation Equalization}
There are many approaches to restoring the color saturation in an image.
It can be adjusted manually with imaging software, or an automatic algorithm
may be used. In the case of the latter, equalizing the saturation levels works
well with most images. It is advantageous to preserve the luminosity of the
image to prevent unrealistic colors.

\section{Examples}

\section{Conclusion}





\end{document}


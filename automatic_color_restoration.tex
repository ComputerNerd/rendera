\documentclass{article}

\title{Automatic Color Restoration}

\author{Joe Davisson} %% \texttt{joe_7@sbcglobal.net}}

\usepackage{amsmath}

\begin{document}
\maketitle

This paper describes simple algorithms for automatically correcting the
color in faded photographs or slides. Since color cast and loss of
saturation are the primary culprits in making an image look bad, I
present some algorithms for reversing these effects.


Color Restore
The dyes used in the photographic process do not fade evenly. Almost any
color cast is possible depending on the storage environment and overall
light exposure. Often there is no way to determine the film or
photograph type, so a more general approach is required.

This color restoration algorithm makes two assumptions: that the image
is faded to some degree, and also has a color cast. I believe it is safe
to assume that undesirable images have both of these traits.

The first pass over the image is used to determine the overall color
cast, which is a simple matter of averaging all the pixel values in the
image. From this averaged value, we can calculate adjustment factors for
each RGB component:

\begin{align}
r_{adjust} &= (256.0 / (256 - r_{average})) / sqrt(256.0 / (r_{average} + 1)); \\
g_{adjust} &= (256.0 / (256 - g_{average})) / sqrt(256.0 / (g_{average} + 1)); \\
b_{adjust} &= (256.0 / (256 - b_{average})) / sqrt(256.0 / (b_{average} + 1));
\end{align}

(The +1 is added to prevent division-by-zero errors in the computer.)

The second pass is to apply these adjustment factors to each pixel in
the source image:

\begin{align}
r &= 255 * pow(r / 255, r_{adjust}); \\
g &= 255 * pow(g / 255, g_{adjust}); \\
b &= 255 * pow(b / 255, b_{adjust});
\end{align}

(These are then clamped to the 0-255 range.)

The power function in the second pass is intended to reverse fading. The
adjustment factors employ a square root to ensure that over-correction
does not occur during this step.


Saturation Equalization
Through some trial-and-error it was found that the saturation loss in
faded images could usually be corrected by pushing the most saturated
color to the maximum, and adjusting the other saturation levels
accordingly -- essentially equalizing the saturation levels in the image.

\end{document}

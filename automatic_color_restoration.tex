\documentclass{article}

\title{Automatic Color Restoration}

\author{Joe Davisson} %% \texttt{joe_7@sbcglobal.net}}

\usepackage{amsmath}

\begin{document}
\maketitle

Some algorithms are presented for automatically correcting the
color and contrast in faded images. Because often the exact nature or cause
of the discoloration is unknown, the problems of color cast, contrast, and
saturation loss are considered individually.

\section{Color Cast Removal}

One way to correct color cast is to simply create adjustment factors by
negating the average of all the pixels in an image:

\[ R_{adjust} = (255 - R_{average}) \]
\[ G_{adjust} = (255 - G_{average}) \]
\[ B_{adjust} = (255 - B_{average}) \]

Then average these with each pixel in the image:

\[ R_{new} = (R_{old} + R_{adjust}) / 2 \]
\[ G_{new} = (G_{old} + G_{adjust}) / 2 \]
\[ B_{new} = (B_{old} + B_{adjust}) / 2 \]

While this serves to reverse the the color cast, the image must be manually
adjusted to restore contrast and saturation. Therefore we aim to do these
adjustments automatically.

\section{Contrast Correction}

The means to restore constrast may be included in the adjustment factors,
combining color cast and contrast correction together. This is accomplished
with the following formula:

\[ R_{adjust} = (256 / (256 - R_{average})) / sqrt(256 / (R_{average} + 1)) \]
\[ G_{adjust} = (256 / (256 - G_{average})) / sqrt(256 / (G_{average} + 1)) \]
\[ B_{adjust} = (256 / (256 - B_{average})) / sqrt(256 / (B_{average} + 1)) \]

These adjustment factors are then applied to the source image as a power factor:

\[ R_{new} = 255 * pow(R_{old} / 255, R_{adjust}) \]
\[ G_{new} = 255 * pow(G_{old} / 255, G_{adjust}) \]
\[ B_{new} = 255 * pow(B_{old} / 255, B_{adjust}) \]

\section{Saturation Equalization}

We make the assumption that the most saturated color in the original faded
image should be pushed to the maximum value, and other saturation levels pushed in relation to that. This is similar to histogram equalization, which is also
well-known and easy to implement.

Histogram equalization is performed on saturation information only, with the option of preserving the luminance of the original image. The latter is important to ensure perceptual integrity.

\end{document}

